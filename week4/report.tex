
\documentclass{article}
\usepackage{amsmath}
\usepackage{graphicx}
\usepackage{listings}
\usepackage{xcolor}

\lstset{
    backgroundcolor=\color{white},
    basicstyle=\footnotesize\ttfamily,
    breaklines=true
}

\title{List Class Testing Program}
\author{Your Name}
\date{\today}

\begin{document}
\maketitle

\section{Introduction}
This document aims to explain the testing program for the custom list class \texttt{List}. The program compares the custom implementation with the standard library \texttt{std::list} to ensure correctness and functionality.

\section{Code Overview}
Below is the main structure of the testing program:

\begin{lstlisting}[language=C++]
#include "List.h"
#include <iostream>
#include <list>
using std::cin;
using std::cout;
using std::endl;
using std::list;
\end{lstlisting}

First, the custom header file \texttt{List.h} and necessary standard library headers are included. The commonly used features from the \texttt{std} namespace are utilized.

\section{Print Function}
A template function \texttt{print} is defined to output the size and elements of the list:

\begin{lstlisting}[language=C++]
template <typename T> void print(T &a) {
  cout << "size: " << a.size() << "  ";
  cout << "elements: ";
  for (auto v : a) {
    cout << v << " ";
  }
  cout << endl;
}
\end{lstlisting}

This function takes a list object and prints its size and all elements.

\section{Main Program}
The main part of the program is as follows:

\begin{lstlisting}[language=C++]
int main() {
  List<int> lst = {1, 2, 3, 4, 5};
  list<int> lst2 = {1, 2, 3, 4, 5};
\end{lstlisting}

First, a custom list \texttt{lst} and a standard library list \texttt{lst2} are created, both initialized with the same elements.

\subsection{Construction Test}
\begin{lstlisting}[language=C++]
cout << "test construct" << endl;
print(lst);
print(lst2);
\end{lstlisting}

Prints the initial state of both lists.

\subsection{Front and Back Element Test}
\begin{lstlisting}[language=C++]
cout << "test front and back" << endl;
cout << lst.front() << " " << lst.back() << endl;
cout << lst2.front() << " " << lst2.back() << endl;
\end{lstlisting}

Verifies the first and last elements of both the custom list and the standard library list.

\subsection{Insert and Delete Test}
\begin{lstlisting}[language=C++]
cout << "test pushback, pushfront, popfront, and popback" << endl;
lst.push_back(10);
lst2.push_back(10);
print(lst);
print(lst2);
\end{lstlisting}

Performs insert and delete operations on both lists, printing the results for comparison.

\subsection{Iterator Test}
\begin{lstlisting}[language=C++]
auto it = lst.begin();
auto it2 = lst2.begin();
++it;
++it2;
cout << "test iterator++: " << endl << *it << " " << *it2 << endl;
\end{lstlisting}

Tests the increment and decrement operations of the iterators to ensure they work correctly in both lists.

\subsection{Copy and Move Semantics Test}
\begin{lstlisting}[language=C++]
cout << "test copy" << endl;
auto lst3 = lst;
auto lst4 = lst2;
\end{lstlisting}

Verifies the copy and move operations of the custom list and the standard library list.

\subsection{Clear Test}
\begin{lstlisting}[language=C++]
lst.clear();
lst2.clear();
cout << "test clear" << endl;
print(lst);
print(lst2);
\end{lstlisting}

Tests the functionality of clearing the lists to ensure they can correctly handle the removal of elements.

\section{Conclusion}
This testing program thoroughly tests various functionalities of the custom list class \texttt{List} and compares them with \texttt{std::list}. These tests verify the correctness and reliability of the custom implementation.

\end{document}

